\documentclass[letterpaper, titlepage, fleqn]{article}

\usepackage[utf8]{inputenc}
\usepackage[slovene]{babel}
\usepackage{amsmath}
\usepackage{amssymb}

\begin{document}

(ii) Naj bo $a \in \mathbb{C}$ in $c \in \mathbb{R}$. Izrazi $E\left( e^{a\left(X+cY\right)}|X\right)$ z $e_a$, $X$ in $c$ ter $E\left(\left(X+cY\right)^{k}|X\right)$ za $k \in \mathbb{N}_0$ z $h_k$, $X$, in $c$. \\
 
Najprej izpostavimo $e^{aX}$ in uporabimo neodvisnost:
\begin{equation*}
\begin{aligned}
E\left( e^{a\left(X+cY\right)}|X\right)=e^{aX}E\left(e^{acY}\right)
\end{aligned}
\end{equation*}
Posebej izračunajmo $E\left(e^{acY}\right)$:
\begin{equation*}
\begin{aligned}
E\left(e^{acY}\right)=&\int_{-\infty}^{\infty}e^{acy}\frac{1}{\sqrt{2\pi}} e^{-\frac{y^2}{2}} dy=\\[8px]
&=\frac{1}{\sqrt{2\pi}}\int_{-\infty}^{\infty}e^{\frac{-\left(y^2-2acy\right)}{2}}dy=\\[8px]
&=\frac{1}{\sqrt{2\pi}}\int_{-\infty}^{\infty}e^{\frac{-\left(\left(y-ac\right)^2-\left(ac\right)^2\right)}{2}}dy=\\[8px]
&=e^\frac{(ac)^2}{2}
\end{aligned}
\end{equation*}
Torej dobimo:
\begin{equation*}
\begin{aligned}
E\left( e^{a\left(X+cY\right)}|X\right)=e^{aX}e^\frac{\left(ac\right)^2}{2}=e^{aX+\frac{a^2c^2}{2}}=e_a\left(X,-c^2\right)
\end{aligned}
\end{equation*}
Za izračun $E\left(\left(X+cY\right)^{k}|X\right)$ uporabimo prejšnji račun in razvijemo $e^{a\left(X+cY\right)}$ v Taylorjevo vrsto:
\begin{equation*}
\begin{aligned}
E\left( e^{a\left(X+cY\right)}|X\right)=E\left(\sum_{i=0}^{\infty}\frac{a^n\left(X+cY\right)^n}{n!}|X\right)=\sum_{i=0}^{\infty}\frac{a^n}{n!}E((X+cY)^n|X) 
\end{aligned}
\end{equation*}
Po definiciji hermitovih polinomov pa velja še $E\left( e^{a\left(X+cY\right)}|X\right)=
\sum_{n=0}^{\infty} \frac{a^n}{n!} h_n\left(X, -c^2\right)$.
Torej dobimo $E\left(\left(X+cY\right)^{n}|X\right)=h_n\left(X,-c^2\right)$.

\end{document}