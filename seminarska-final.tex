\documentclass[ letterpaper, titlepage, fleqn]{article}

\usepackage[utf8]{inputenc}
\usepackage[slovene]{babel}
\usepackage[margin=60px]{geometry}
\usepackage{amsmath}
\usepackage{amssymb}

\begin{document}
\title{Seminar \\ Hermitovi polinomi in normalna porazdelitev}
\author{Nejc Ševerkar \& Tjaša Renko \\ mentor Janez Bernik}
\date{\today}
\maketitle

\begin{abstract}
\begin{center}
V tej nalogi bova predstavila povezavo med hermitovimi polinomi
in standardno normalno porazdelitvijo.
\end{center}
\end{abstract}

\section*{Naloga}
Hermitovi polinomi $(h_n)_{n \in \mathbb{N}}$ dveh realnih spremenljivk
so definirani preko relacije \\[10px]
$e_a(x, t) := e^{ax -\frac{a^2t}{2}} = \sum_{n=0}^{\infty} \frac{a^n}{n!} h_n(x, t), 
\quad {x, t} \subset \mathbb{R}, a \in \mathbb{C}$ \\[10px]
Naj bosta $X \sim N(0, 1)$ in $Y \sim N(0, 1)$ neodvisni standardni normalni spremenljivki. \\[10px]
(i) Izračunaj $E[e_a(X,1) e_b(X,1)] (\{a,b\} \in \mathbb{C}) $ in izpelji izraz za $ E[h_n(X,1) h_m(X,1)] \text{ } (\{m,n\}\in ~ \mathbb{N} $).\\

{\setlength{\mathindent}{0cm}
\begin{equation*}
\begin{aligned}
E\left(e_a\left(X, t\right)e_b\left(X, t\right)\right) &= E\left(g\left(X\right)\right) = \int_{-\infty}^{\infty}g\left(x\right)f_X\left(x\right)dx = \\[8px]
&= \int_{-\infty}^{\infty} e^{-\frac{t}{2}(a^2 + b^2)} e^{x \left(a + b\right)} \frac{1}{\sqrt{2\pi}} e^{-\frac{x^2}{2}} dx \\[8px]
&= e^{-\frac{t}{2}\left(a^2 + b^2\right)} \int_{-\infty}^{\infty} \frac{1}{\sqrt{2\pi}} e^{-\frac{x^2}{2} + x \left(a + b\right)} dx \\[8px]
& = e^{-\frac{t}{2} \left(a^2 + b^2\right)} \int_{-\infty}^{\infty} \frac{1}{\sqrt{2\pi}} e^{-\frac{1}{2} \left(x^2 - 2x \left(a + b\right)\right)} dx \\[8px]
&= e^{-\frac{t}{2} \left(a^2 + b^2\right)} \int_{-\infty}^{\infty} \frac{1}{\sqrt{2\pi}} e^{-\frac{1}{2} \left(\left(x - \left(a + b\right)\right)^2 -\left (a + b\right)^2\right)} dx \\[8px]
& = e^{-\frac{t}{2} \left(a^2 + b^2\right)} e^{\frac{1}{2} \left(a + b\right)^2} \int_{-\infty}^{\infty} \frac{1}{\sqrt{2\pi}} e^{-\frac{1}{2} \left(x - \left(a + b\right)\right)^2} dx \\[8px]
& = e^{\frac{1}{2} \left(\left(a + b\right)^2 - t \left(a^2 + b^2\right)\right)} \int_{-\infty}^{\infty} f_Z(x) dx \quad (Z \sim N(a + b, 1)) \\[8px]
& = e^{\frac{1}{2} \left(\left(a + b\right)^2 - t \left(a^2 + b^2\right)\right)}
\end{aligned}
\end{equation*}

\begin{equation*}
E\left(e_a\left(X, 1\right) e_b\left(X, 1\right)\right) = e^{ab} = \sum_{i=0}^{\infty} \frac{(ab)^i}{i!} \\[8px]
E\left(e_a\left(X, 1\right) e_b\left(X, 1\right)\right) = E\left(\sum_{n=0}^{\infty} \sum_{m=0}^{\infty} \frac{a^n b^m}{n! m!} h_n(x, 1) h_m(x, 1)\right) \\[8px]
\end{equation*}
Vidimo, da enakost sledi tedaj, ko sta za $ \forall n \neq m : h_n(x, 1) h_m(x, 1) = 0 $ in za $ \forall n = m : h_n(x, 1) h_m(x, 1) = n! $ \\[10px]
\[
$$
E(h_n(x, 1) h_m(x, 1)) = \int_{-\infty}^{\infty} h_n(x, 1) h_m(x, 1) f_X(x) dx  =
\left\{
\begin{array}{lr}
 n! & \text{za } m = n \\
 0 & \text{za } m \neq n 
 \end{array} 
 \right.
$$
\]
\pagebreak


\noindent (ii) Naj bo $a \in \mathbb{C}$ in $c \in \mathbb{R}$. \\
\indent Izrazi $E\left( e^{a\left(X+cY\right)}|X\right)$ z $e_a$, $X$ in $c$ ter $E\left(\left(X+cY\right)^{k}|X\right)$ za $k \in \mathbb{N}_0$ z $h_k$, $X$, in $c$.
\\[8px]
 
\noindent Najprej izpostavimo $e^{aX}$ in uporabimo neodvisnost:
\begin{equation*}
\begin{aligned}
E\left( e^{a\left(X+cY\right)}|X\right)=e^{aX}E\left(e^{acY}\right)
\end{aligned}
\end{equation*}
Posebej izračunajmo $E\left(e^{acY}\right)$:
\begin{equation*}
\begin{aligned}
E\left(e^{acY}\right)&=\int_{-\infty}^{\infty}e^{acy}\frac{1}{\sqrt{2\pi}} e^{-\frac{y^2}{2}} dy=\\[8px]
&=\frac{1}{\sqrt{2\pi}}\int_{-\infty}^{\infty}e^{\frac{-\left(y^2-2acy\right)}{2}}dy=\\[8px]
&=\frac{1}{\sqrt{2\pi}}\int_{-\infty}^{\infty}e^{\frac{-\left(\left(y-ac\right)^2-\left(ac\right)^2\right)}{2}}dy=\\[8px]
&=e^\frac{(ac)^2}{2}
\end{aligned}
\end{equation*}
Torej dobimo:
\begin{equation*}
\begin{aligned}
E\left( e^{a\left(X+cY\right)}|X\right)=e^{aX}e^\frac{\left(ac\right)^2}{2}=e^{aX+\frac{a^2c^2}{2}}=e_a\left(X,-c^2\right)
\end{aligned}
\end{equation*}
Za izračun $E\left(\left(X+cY\right)^{k}|X\right)$ uporabimo prejšnji račun in razvijemo $e^{a\left(X+cY\right)}$ v Taylorjevo vrsto:
\begin{equation*}
\begin{aligned}
E\left( e^{a\left(X+cY\right)}|X\right)=E\left(\sum_{i=0}^{\infty}\frac{a^n\left(X+cY\right)^n}{n!}|X\right)=\sum_{i=0}^{\infty}\frac{a^n}{n!}E((X+cY)^n|X) 
\end{aligned}
\end{equation*}
Po definiciji hermitovih polinomov pa velja še $E\left( e^{a\left(X+cY\right)}|X\right)=
\sum_{n=0}^{\infty} \frac{a^n}{n!} h_n\left(X, -c^2\right)$. \\
Dobimo torej $E\left(\left(X+cY\right)^{n}|X\right)=h_n\left(X,-c^2\right)$.

\end{document}