\documentclass[ letterpaper, titlepage, fleqn]{article}

\usepackage[utf8]{inputenc}
\usepackage[slovene]{babel}
\usepackage[margin=60px]{geometry}
\usepackage{amsmath}
\usepackage{amssymb}
\usepackage{enumerate}
\setlength\parindent{0pt}

\begin{document}
\title{Seminar \\ Hermitovi polinomi in normalna porazdelitev}
\author{Nejc Ševerkar \& Tjaša Renko \\ mentor Janez Bernik}
\date{\today}
\maketitle

\begin{abstract}
\begin{center}
V tej nalogi bova predstavila povezavo med hermitovimi polinomi
in standardno normalno porazdelitvijo.
\end{center}
\end{abstract}

\section*{Naloga}
Hermitovi polinomi $(h_n)_{n \in \mathbb{N}}$ dveh realnih spremenljivk
so definirani preko relacije \\[10px]
$e_a(x, t) := e^{ax -\frac{a^2t}{2}} = \sum_{n=0}^{\infty} \frac{a^n}{n!} h_n(x, t), 
\quad {x, t} \subset \mathbb{R}, a \in \mathbb{C}$.\\[10px]
Naj bosta $X \sim N(0, 1)$ in $Y \sim N(0, 1)$ neodvisni standardni normalni spremenljivki.

\begin{enumerate}[(i)]
\item Izračunaj $ \mathbb{E}[e_a(X,1) e_b(X,1)] (\{a,b\} \in \mathbb{C})$ 
in izpelji izraz za $\mathbb{E}[h_n(X,1) h_m(X,1)] \text{ } (\{m,n\}\in ~ \mathbb{N} $).

<<<<<<< HEAD
Pri prvem delu moramo izračunati $\mathbb{E}\left(g\left(X\right)\right)$, kjer smo označili $g\left(x\right) = e_a\left(x, t\right) e_b\left(x, t\right)$ .
Poenostavimo najprej $g(x)$, torej
$$
g(x) = e_a(x, 1) e_b(x, 1) = e^{ax - \frac{a^2}{2}}  e^{bx - \frac{b^2}{2}} = e^{-\frac{a^2 + b^2}{2}} e^{x(a + b)}
$$
Vstavimo rezultat v začetni problem in upoštevamo izrek o matematičnem upanju transformacije slučajne sremenljivke
$$
\mathbb{E}\left(e_a\left(X, t\right)e_b\left(X, t\right)\right) =
\mathbb{E}\left(g\left(X\right)\right) = \int_{-\infty}^{\infty}g\left(x\right)f_X\left(x\right)dx =
e^{-\frac{a^2 + b^2}{2}} \int_{-\infty}^{\infty}  e^{x(a + b)} \frac{1}{\sqrt{2\pi}} e^{-\frac{x^2}{2}} dx
$$
Sedaj del integranda v eksponentu $e$ lahko dopolnimo do popolnega kvadrata,z namenom preoblikovanja integranda v obliko,
ki bo skladna z gostoto premaknjene normalne porazdelitvije.
=======
Pri prvem delu moramo izračunati $\mathbb{E}\left(g\left(X\right)\right)$, kjer smo označili $g\left(x\right) = e_a\left(x, t\right) e_b\left(x, t\right)$ . Za izračun upoštevamo izrek o matematičnem upanju transformacije slučajne spremenljivke, ki je razviden v računu:

{\setlength{\mathindent}{0cm}}
>>>>>>> cd1e0e255fd81aebb9fa6d1b280af3c4f4acd0d2
\begin{equation*}
\begin{aligned}
&= e^{-\frac{1}{2} (a^2 + b^2)} \int_{-\infty}^{\infty} \frac{1}{\sqrt{2\pi}} e^{-\frac{1}{2} (x^2 - 2x (a + b))} dx \\
&= e^{-\frac{1}{2} (a^2 + b^2)} \int_{-\infty}^{\infty} \frac{1}{\sqrt{2\pi}} e^{-\frac{1}{2} ((x - (a + b))^2 - (a + b)^2)} dx  \\
&= e^{\frac{1}{2}((a + b)^2 - (a^2 + b^2))} \int_{-\infty}^{\infty} \frac{1}{\sqrt{2\pi}} e^{-\frac{1}{2} (x - (a + b))^2} dx  \\
& = e^{ab} \int_{-\infty}^{\infty} f_Z(x) dx \qquad (Z \sim N(a + b, 1)) \\
& = e^{ab}.
\end{aligned}
\end{equation*}

Dobimo torej naslednji zvezi:
\begin{equation*}
\begin{aligned}
\mathbb{E}\left(e_a\left(X, 1\right) e_b\left(X, 1\right)\right) &= e^{ab} = \sum_{i=0}^{\infty} \frac{(ab)^i}{i!}, \\[8px]
\mathbb{E}\left(e_a\left(X, 1\right) e_b\left(X, 1\right)\right) &= \mathbb{E}\left(\sum_{n=0}^{\infty} \sum_{m=0}^{\infty} \frac{a^n b^m}{n! m!} h_n(x, 1) h_m(x, 1)\right)  =
\sum_{n=0}^{\infty} \sum_{m=0}^{\infty} \frac{a^n b^m}{n! m!} \mathbb{E}\left( h_n(x, 1) h_m(x, 1)\right).
\\[8px]
\end{aligned}
\end{equation*}

$
\text{Vidimo, da enakost med vrstama velja, ko }
\mathbb{E}\left(h_n\left(X, 1\right) h_m\left(X, 1\right)\right)  =
\left\{
\begin{array}{lr}
n! & \text{za } m = n \\
0 & \text{za } m \neq n 
\end{array} 
.\right.
$

Med računom smo uporabili izrek o zamenjavi vrstnega reda pričakovane vrednosti in vrste slučajnih spremenljivk, za kar smo potrebovali pogoj 
absolutne konvergence $\sum_{n=0}^{\infty} \sum_{m=0}^{\infty} \frac{a^n b^m}{n! m!}  h_n(|X|, 1) h_m(|X|, 1)$, 
ki sledi iz enakosti tega in svoje limite $e_a\left(|X|, 1\right) e_b\left(|X|, 1\right)$.

Za naš primer lahko dokažemo, da velja
$$
\mathbb{E}\left(\sum_{n=0}^{\infty} \frac{a^n}{n!} h_n(X, 1)\right) = \sum_{n=0}^{\infty} \frac{a^n}{n!} \mathbb{E}(h_n(X, 1)),
$$
kar je ekvivalentno 
$$
\int_{-\infty}^{\infty} \sum_{n=0}^{\infty} \frac{a^n}{n!} h_n(x, 1) f_X(x) dx =
\sum_{n=0}^{\infty} \frac{a^n}{n!}\int_{-\infty}^{\infty} h_n(x, 1) f_X(x) dx
$$.
Poglejmo si razliko

\begin{equation*}
\begin{aligned}
<<<<<<< HEAD
&\left|\int_{-\infty}^{\infty} \sum_{n=0}^{\infty} \frac{a^n}{n!} h_n(x, 1) f_X(x) dx - 
\sum_{n=0}^{M} \frac{a^n}{n!} \int_{-\infty}^{\infty} h_n(x, 1) f_X(x)dx\right| = \\
&\left|\int_{-\infty}^{\infty} \left( \sum_{n=0}^{\infty} \frac{a^n}{n!} - \sum_{n=0}^{M} \frac{a^n}{n!} h_n(x, 1)\right) f_X(x) dx\right| \leq \\
&\int_{-\infty}^{\infty} \left|\sum_{n=0}^{\infty} \frac{a^n}{n!} h_n(x, 1) - \sum_{n=0}^{M} \frac{a^n}{n!} h_n(x, 1)\right| f_X(x) dx <  \\
&\int_{-\infty}^{\infty} \epsilon f_X(x) dx = \epsilon
=======
&|\int_{-\infty}^{\infty} \sum_{n=0}^{\infty} \frac{a^n}{n!} h_n(x, 1) f_X(x) dx - 
\sum_{n=0}^{M} \frac{a^n}{n!} \int_{-\infty}^{\infty} h_n(x, 1) f_X(x)dx| = \\
&|\int_{-\infty}^{\infty} \left( \sum_{n=0}^{\infty} \frac{a^n}{n!} - \sum_{n=0}^{M} \frac{a^n}{n!} h_n(x, 1)\right) f_X(x) dx| \leq \\
&\int_{-\infty}^{\infty} |\sum_{n=0}^{\infty} \frac{a^n}{n!} h_n(x, 1) - \sum_{n=0}^{M} \frac{a^n}{n!} h_n(x, 1)| f_X(x) dx <  \\
&\int_{-\infty}^{\infty} \epsilon f_X(x) dx = \epsilon.
>>>>>>> cd1e0e255fd81aebb9fa6d1b280af3c4f4acd0d2
\end{aligned}
\end{equation*}

Upoštevali smo, da $\sum_{n=0}^{\infty} \frac{a^n}{n!} h_n(x, 1)$ konvergira enakomerno na zaprtih intervalih $[-A, A]$, 
za poljuben $A > 0$, torej lahko njegovo in limitno razliko omejimo z $\epsilon$ na poljubno velikem intervalu,
torej tudi v limiti, ki jo posplošeni integral zahteva.
Iz tega sledi, da sta razliki za dovolj velik $M \in \mathbb{N}$ poljubno blizu.\bigskip


\item Naj bo $a \in \mathbb{C}$ in $c \in \mathbb{R}$. Izrazi $\mathbb{E}\left( e^{a\left(X+cY\right)}|X\right)$ z $e_a$, $X$ in $c$ ter $\mathbb{E}\left(\left(X+cY\right)^{k}|X\right)$ za $k \in \mathbb{N}_0$ z $h_k$, $X$, in $c$.
 
Najprej upoštevamo lastnosti pogojnega matematičnega upanja in uporabimo neodvisnost. Tako dobimo:
\begin{equation*}
\begin{aligned}
\mathbb{E}\left( e^{a\left(X+cY\right)}|X\right)=\mathbb{E}\left( e^{aX}e^{acY}|X\right)=e^{aX}\mathbb{E}\left(e^{acY}\right).
\end{aligned}
\end{equation*}
Posebej izračunajmo $\mathbb{E}\left(e^{acY}\right)$:
\begin{equation*}
\begin{aligned}
\mathbb{E}\left(e^{acY}\right)&=\int_{-\infty}^{\infty}e^{acy}\frac{1}{\sqrt{2\pi}} e^{-\frac{y^2}{2}} dy=\\[8px]
&=\frac{1}{\sqrt{2\pi}}\int_{-\infty}^{\infty}e^{\frac{-\left(y^2-2acy\right)}{2}}dy=\\[8px]
&=\frac{1}{\sqrt{2\pi}}\int_{-\infty}^{\infty}e^{\frac{-\left(\left(y-ac\right)^2-\left(ac\right)^2\right)}{2}}dy=\\[8px]
&=e^\frac{(ac)^2}{2}.
\end{aligned}
\end{equation*}
Torej dobimo:
\begin{equation*}
\begin{aligned}
\mathbb{E}\left( e^{a\left(X+cY\right)}|X\right)=e^{aX}e^\frac{\left(ac\right)^2}{2}=e^{aX+\frac{a^2c^2}{2}}=e_a\left(X,-c^2\right).
\end{aligned}
\end{equation*}
Za izračun $\mathbb{E}\left(\left(X+cY\right)^{k}|X\right)$ uporabimo prejšnji račun in razvijemo $e^{a\left(X+cY\right)}$ v Taylorjevo vrsto:
\begin{equation*}
\begin{aligned}
\mathbb{E}\left( e^{a\left(X+cY\right)}|X\right)=\mathbb{E}\left(\sum_{n=0}^{\infty}\frac{a^n\left(X+cY\right)^n}{n!}|X\right)=\sum_{n=0}^{\infty}\frac{a^n}{n!}\mathbb{E}((X+cY)^n|X). 
\end{aligned}
\end{equation*}
Po definiciji hermitovih polinomov pa velja še $\mathbb{E}\left( e^{a\left(X+cY\right)}|X\right)=
\sum_{n=0}^{\infty} \frac{a^n}{n!} h_n\left(X, -c^2\right)$. \\
Dobimo torej $\mathbb{E}\left(\left(X+cY\right)^{n}|X\right)=h_n\left(X,-c^2\right)$.
\end{enumerate}
\end{document}