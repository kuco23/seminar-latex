\documentclass[letterpaper, notitlepage]{article}

\usepackage[utf8]{inputenc}
\usepackage[margin=1cm]{geometry}
\usepackage{amsmath}
\usepackage{amssymb}

\begin{document}

\title{Hermitovi polinomi in normalna porazdelitev}
\author{Nejc Ševerkar \& Tjaša Redek}
\date{\today}
\maketitle

\begin{abstract}
V tej nalogi bova s Tjašusom predstavila
povezavo med Hermitovimi polinomi in normalno porazdellitvijo.
Na koncu bova sprejela tudi kakšno vprašanje.
Če bo najino ne poznanje odgovora na to vprašanje
rezultiralo v nižjo oceno bova izvedela kje oseba, ki
je to vprašanje zastavila živi.
\end{abstract}
\pagebreak

\section*{Naloga}
Hermitovi polinomi $(h_n)_{n \in \mathbb{N}}$ dveh realniih spremenljivk
so definirani preko relacije \\

\begin{center}
$e_n(x, t) := e^{ax -\frac{a^2t}{2}} = \sum_{n=0}^{\infty} \frac{a^n}{n!} h_n(x, t), 
\quad {x, t} \subset \mathbb{R}, a \in \mathbb{C}$
\end{center}

Naj bosta $X \sim N(0, 1)$ in $Y \sim N(0, 1)$ neodvisni standardni normalni spremenljivki


\end{document}